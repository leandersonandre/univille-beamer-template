\documentclass{beamer}
\usepackage[utf8]{inputenc}
\usepackage[T1]{fontenc}
\usepackage[portuguese]{babel}
\usepackage{listings, xcolor}
\usepackage{lstautogobble}
\usepackage{colortbl} 
 \usepackage{booktabs}
\usetheme{Madrid}
\usepackage{ragged2e}
\usepackage{soul}          % para \hl (highlight)
\usepackage[normalem]{ulem}
\beamertemplatenavigationsymbolsempty

\usepackage{tikz}
\usetikzlibrary{shapes.geometric, arrows}


\setbeamertemplate{title page}[default][colsep=-0bp,rounded=false]
\setbeamertemplate{frametitle}[default][colsep=-4bp,rounded=false,shadow=false]
\setbeamertemplate{blocks}[default]
\setbeamertemplate{headline}[shadow=false]
\setbeamertemplate{subsection in head}[shadow=false]
\setbeamertemplate{section in head}[shadow=false]
\setbeamertemplate{beamercolorbox}[shadow=false]

% \usepackage{beamerthemesplit} // Activate for custom appearance
\definecolor{ao(english)}{rgb}{0.0, 0.5, 0.0}
\setbeamercolor{structure}{fg=ao(english)}
\definecolor{darkgreen}{rgb}{0,.5,0}
\definecolor{codegray}{rgb}{0.5,0.5,0.5}

\renewcommand{\lstlistingname}{Algoritmo}
\lstdefinestyle{myJavaStyle}{
  language=Java,
  numbers=left,
  firstnumber=1,
  stepnumber=1,
  numbersep=10pt,
  tabsize=4,
  showspaces=false,
  showstringspaces=false,
  keywordstyle=\color{blue}\ttfamily,
  stringstyle=\color{red}\ttfamily,
  commentstyle=\color{darkgreen}\ttfamily,
  basicstyle=\ttfamily,
  breaklines=true,
  autogobble=true,
  backgroundcolor=\color{gray!10!white},    
  numbersep=5pt, 
  numberstyle=\tiny\color{codegray},  
  escapeinside={(*@}{@*)},
  literate=
    {á}{{\'a}}1
    {à}{{\`a}}1
    {ã}{{\~a}}1
    {â}{{\^a}}1
    {é}{{\'e}}1
    {ê}{{\^e}}1
    {í}{{\'i}}1
    {ó}{{\'o}}1
    {ô}{{\^o}}1
    {õ}{{\~o}}1
    {ú}{{\'u}}1
    {ç}{{\c{c}}}1
    {Á}{{\'A}}1
    {À}{{\`A}}1
    {Ã}{{\~A}}1
    {Â}{{\^A}}1
    {É}{{\'E}}1
    {Ê}{{\^E}}1
    {Í}{{\'I}}1
    {Ó}{{\'O}}1
    {Ô}{{\^O}}1
    {Õ}{{\~O}}1
    {Ú}{{\'U}}1
    {Ç}{{\c{C}}}1
}
\lstset{basicstyle=footnotesize,style=myJavaStyle}

\setbeamertemplate{endpage}{%
    \begin{frame}
        \begin{figure}[h]
    \centering
    \includegraphics[width=0.5\textwidth]{univille_logo.png}
   \end{figure}
   \begin{center}
   Prof. MSc. Leanderson André\\
   leandersonandre@univille.br
   \end{center}
   \end{frame}
}

\title[Minha Apresentação no Rodapé]{Título da Minha Apresentação}
\subtitle{Meu Subtítulo da minha Apresentação}
\author[]{Prof. MSc. Leanderson André}
\institute[]{\inst{} Universidade da Região de Joinville }
\date{\today}

\begin{document}

\frame{\titlepage}

\section[Outline]{}
\frame{
\frametitle{Agenda}
\tableofcontents
}

%\section{Introduction}
%\subsection{Overview of the Beamer Class}


\section{Beamer}
\begin{frame}[fragile]{Beamer}

\textbf{Beamer} é uma classe de documento do LaTeX usada para criar apresentações em slides de alta qualidade tipográfica, semelhante ao PowerPoint ou Keynote, mas com foco em clareza, elegância e controle completo do conteúdo via código.



Tutorial em \url{https://www.overleaf.com/learn/latex/Beamer_Presentations%3A_A_Tutorial_for_Beginners_(Part_1)—Getting_Started}.

\end{frame}


\subsection{Blocos e listas}
\begin{frame}[fragile]{Blocos}

\begin{block}{Título do bloco}
Conteúto do bloco.

\end{block}



\begin{alertblock}{Título do bloco de alerta}
Conteúto do bloco.

\end{alertblock}

\begin{itemize}
\item Item da lista
\item Item da lista
\item Item da lista
\end{itemize}

\end{frame}


\subsection{Elementos textuais}
\begin{frame}[fragile]{Elementos textuais}

\textbf{Negrito} \\
\textit{Itálico} \\
\underline{Sublinhado} \\
Texto com subscrito: $H_2O$ \\
Texto com sobrescrito: $x^2 + y^2 = z^2$ \\
Texto \sout{riscado} \\
Texto normal e \colorbox{yellow}{texto destacado com fundo amarelo} no meio da frase. \\
Texto em \texttt{monoespaçado}

\end{frame}

\begin{frame}{Exemplo de colunas}

\begin{columns}[T] % [T] alinha ao topo
  \column{0.5\textwidth}
  \textbf{Coluna 1:}
  \begin{itemize}
    \item Item 1
    \item Item 2
    \item Item 3
  \end{itemize}

  \column{0.5\textwidth}
  \textbf{Coluna 2:}
  \begin{itemize}
    \item Outro item A
    \item Outro item B
  \end{itemize}
  
\end{columns}

\end{frame}

\begin{frame}{Blocos lado a lado}

\begin{columns}
  \column{0.48\textwidth}
  \begin{block}{Bloco 1}
  Conteúdo do bloco 1.
  \end{block}

  \column{0.48\textwidth}
  \begin{alertblock}{Bloco 2}
  Conteúdo do bloco 2.
  \end{alertblock}
\end{columns}

\end{frame}


\begin{frame}{Aparecer itens progressivamente}

\begin{itemize}
  \item<1-> Primeiro item
  \item<2-> Segundo item
  \item<3-> Terceiro item
\end{itemize}

\end{frame}

\section{Figuras}
\begin{frame}[fragile]{Exemplo de Figura}

Utilize o pacote figure. Informe o nome e o tamanho da imagem.

Tutorial em \url{https://www.overleaf.com/learn/latex/Inserting_Images}.

\begin{figure}[h]
\centering
\includegraphics[width=0.5\textwidth]{univille_logo.png}
\caption{Título da Figura}
\end{figure}

\end{frame}

\section{Algoritmos}
\begin{frame}[fragile]{Exemplo de código}\include{bibliografia}

No pacote de algoritmos (lstlisting), o frame deve ter o comando [fragile].
É possível colocar o hightlight conforme a linguagem de programação.
Tutorial em \url{https://www.overleaf.com/learn/latex/Code_listing}.

\begin{lstlisting}[caption=Percorrer por todas as posições da matriz]
int m = { {1, 2, 3 }, {4, 5, 6}, {7, 8, 9}};
// Estrutura de repetição clássica
// Necessário utilizar a posição no vetor
for(int i =0; i < m.length; i++){
   for(int j =0; j < m[i].length; j++){
     System.out.println(m[i][j]);
   }
}
\end{lstlisting}

\end{frame}


\section{Tabelas}
\begin{frame}[fragile]{Exemplo de tabela}

No site \url{https://www.tablesgenerator.com} possui um gerador de tabelas em latex.

\begin{table}[]
\centering
\begin{tabular}{llll}
\hline
1 & 2 & 3 & 4             \\
5 & 6 & 7 & 8             \\
9 & 10 & 11 & 12             \\
13 & 14 & 15 & 16             \\ \hline
\end{tabular}
\caption{Minha tabela}
\end{table}


\end{frame}

\begin{frame}{Tabela com booktabs}

\centering
\begin{tabular}{lcr}
\toprule
Nome & Idade & Nota \\
\midrule
Ana & 22 & 8.5 \\
Bruno & 25 & 9.0 \\
Carlos & 20 & 7.8 \\
\bottomrule
\end{tabular}

\end{frame}



\section{Desenho com TikZ}
\begin{frame}
\frametitle{Desenho simples com TikZ}

Tutorial em \url{https://www.overleaf.com/learn/latex/TikZ_package}. Peça ao chatgtp gerar o código.

\begin{center}
\begin{tikzpicture}
  % Desenha um círculo azul
  \draw[blue, thick] (0,0) circle (2cm);

  % Desenha um quadrado vermelho
  \draw[red, thick] (-1,-1) rectangle (1,1);

  % Texto no centro
  \node at (0,0) {TikZ};
\end{tikzpicture}
\end{center}

\end{frame}


\section{Referências}
\begin{frame}[fragile]{Como utilizar as Referências}

Para citar um artigo \cite{bloch2017effective}, utilize o comando \textbackslash cite. Confere o arquivo biliografia.bib. Neste arquivo você adiciona as referências. Utilize o google scholar para obter a referência no formato bibtex ou peça ao chatgpt.

\begin{block}{Prompt}
Gere a bibliografia em formato bibtex do \textcolor{red}{ livro effective java de joshua bosh}. Me dê apenas a bibliografia.	
\end{block}

Por padrão, é apresentado apenas as referências citadas. Caso desejar utilize o comando \textbackslash nocite\{*\} para mostrar todas as referências no arquivo. Confira o código fonte do próximo slide.


\end{frame}

\begin{frame}{Citação}

\begin{quote}
``Programs must be written for people to read, and only incidentally for machines to execute.'' \\
\hfill --- Harold Abelson, \textit{Structure and Interpretation of Computer Programs}
\end{quote}

\end{frame}


\begin{frame}[fragile]{Referências}

        \bibliographystyle{amsalpha}
        \bibliography{bibliografia.bib}
        \nocite{*}

\end{frame}

\usebeamertemplate{endpage}

\end{document}
